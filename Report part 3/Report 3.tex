
\documentclass[11pt]{article}
\title{\textbf{Smart Classroom Prototype: Implementation Overview}}
\author{Anou Oussama - El Hazal Salma - Lakhiri Rim}
\date{January 6, 2024}
\usepackage{fullpage}
\usepackage{graphicx}

\begin{document}
	\maketitle
	
	\section{Introduction}
	\hspace{1cm} The implementation phase focuses on integrating face recognition capabilities into the Smart Classroom prototype using Python, an API, and a PC's camera. This section outlines the key steps and components involved in the face recognition implementation, emphasizing that this is a prototype to test and validate the feasibility of the proposed solution.
	
	\hspace{1cm} \subsection{Overview}
	\hspace{1cm} Face recognition is implemented to enhance the security and functionality of the Smart Classroom prototype. In this project, we concentrate on using a PC's camera for facial recognition, providing a cost-effective alternative. An external API is utilized to handle the intricate aspects of facial recognition algorithms, allowing for accurate and efficient identification of individuals.
	
	\hspace{1cm} \subsection{Components and Workflow}
	 \subsubsection{Camera Setup}
	\hspace{1cm} \hspace{1cm} The camera installed on the classroom ceiling plays a crucial role in capturing video sessions. It continuously records the classroom activities and sends the footage to the Raspberry Pi for real-time face recognition.
	
	\hspace{1cm} \hspace{1cm} \subsubsection{Raspberry Pi Integration}
	\hspace{1cm} \hspace{1cm} The Raspberry Pi, positioned on the door frame, acts as the central hub for face recognition in the prototype. It communicates with the camera, processes the video feed, and interacts with the external face recognition API.
	
	\hspace{1cm} \hspace{1cm} \subsubsection{External Face Recognition API}
	\hspace{1cm} \hspace{1cm} The external API is responsible for the complex task of face recognition. It leverages advanced algorithms to analyze facial features, compare them against a pre-existing database, and determine the identity of individuals in the classroom.
	
	\hspace{1cm} \subsection{Python Scripting}
	\hspace{1cm} The implementation involves the development of Python scripts to facilitate communication between the PC, camera, and external API. These scripts handle data transfer, error handling, and the overall orchestration of the face recognition process.
	
	\hspace{1cm} \subsection{Recording Lectures}
	\hspace{1cm} The Smart Classroom prototype, now equipped with face recognition capabilities, extends its functionality to record lectures. When a recognized face enters the classroom, the system initiates the recording of the session, creating a valuable resource for later review or archival purposes.
	
	\hspace{1cm} \subsection{Challenges and Solutions}
	\hspace{1cm} Throughout the implementation phase of the prototype, challenges such as lighting conditions, varying facial expressions, and database management may arise. Robust error-handling mechanisms and continuous optimization of the face recognition algorithm are implemented to address these challenges and enhance the reliability of the prototype.
	
	\hspace{1cm} The successful implementation of face recognition in the Smart Classroom prototype significantly contributes to its functionality, serving as a foundation for further development and refinement.
	
	\pagebreak
	
\end{document}