\documentclass{article}

\usepackage{geometry}
\usepackage{lipsum}
\usepackage{enumitem}
\usepackage{ragged2e}
\usepackage{titlesec}
\usepackage{hyperref}

\titleformat{\section}{\LARGE\bfseries}{\thesection}{1em}{}
\titleformat{\subsection}{\Large\bfseries}{\thesubsection}{1em}{}
\newgeometry{
	top=2cm,
	bottom=2cm,
	outer=2cm,
	inner=2cm,
}

\begin{document}
	
	\begin{center}
		\vspace*{2cm}
		{\Huge\bfseries Fiche de projet : Smart Classroom} \\
		\vspace{0.5cm}
		\textit{Porteur de projet:} Anou Oussama - El-hazal Salma - Lakhiri Rim\\
		\vspace{1.5cm}
	\end{center}
	
	\section{Introduction}
	
	\subsection*{Objectifs du projet :}
	\begin{itemize}
		\item Automatiser la prise de présence des étudiants en utilisant la reconnaissance faciale à l'entrée des cours.
		\item Avoir une base de données qui facilite l'évaluation de chaque étudiant.
		\item Simplifier le processus de gestion des absences.
	\end{itemize}
	
	\subsection*{Motivation du projet :}
	\begin{itemize}
		\item Améliorer l'efficacité : En automatisant le processus de prise de présence à l'aide de la reconnaissance faciale, le projet vise à optimiser la gestion du temps en classe et à réduire les tâches administratives manuelles liées au suivi des absences.
		\item Renforcer la sécurité : En utilisant des technologies de reconnaissance faciale, le projet permet de contrôler l'accès aux salles de classe et de s'assurer que seuls les étudiants autorisés peuvent y entrer.
		\item Faciliter l'évaluation : La création d'une base de données contenant les empreintes faciales des étudiants ainsi que leurs informations personnelles facilite l'évaluation individuelle de chaque étudiant.
		\item Favoriser la numérisation : Le projet vise à promouvoir la transformation numérique des environnements d'apprentissage en intégrant des technologies avancées.
	\end{itemize}
	
	\section{Description du projet}
	Le projet \textit{Smart Classroom} vise à metttr en place les services suivants:
	\begin{itemize}
		\item Pointage de présence avec une caméra placée à l'entrée de la salle.
		\item Une autre caméra pour enregistrer la séance et prendre des captures pour le tableau.
		\item Permettre aux membres de l'école d'accéder à leur espace personnel dans un site web dédié au suivi des absences en fonction de leurs autorisations. Les étudiants ont la possibilité de consulter leur profil afin de connaître leur statut de présence dans toutes les matières. Les enseignants, quant à eux, ont le droit de supprimer ou de modifier le statut de présence des étudiants.
	\end{itemize}
	
	\section{Besoins matériels et logiciels}
	\subsection{Besoin matériel :}
	\begin{itemize}
		\item Raspberry Pi Camera Module
		\item LED lights
		\item Buzzer
	\end{itemize}
	
	\subsection{Logiciels nécessaires pour réaliser le projet}
	\begin{itemize}
		\item IDE: VsCode, PyCharm, Thonny
		\item Python Frameworks: Django, OpenCV
		\item Web tools: HTML, CSS, JavaScript
		\item OS: Raspbian
	\end{itemize}
	
	\section{Conclusion}
	En conclusion, ce projet marque une première étape vers un vaste processus de digitalisation qui s'inscrit dans le cadre du programme scolaire Go Digital de l’INPT. Il vise à numériser certains aspects et tâches quotidiennes des étudiants et des enseignants, facilitant ainsi le processus d'apprentissage au sein de notre école.
	
	\section{Références Bibliographiques}
	\href{https://www.raspberrypi.com/}{https://www.raspberrypi.com/}\\
	\href{https://towardsdatascience.com/real-time-face-recognition-an-end-to-end-project-b738bb0f7348}{https://towardsdatascience.com/}
	
\end{document}